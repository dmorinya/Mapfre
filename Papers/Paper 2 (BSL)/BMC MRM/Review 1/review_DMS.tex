% LaTeX rebuttal letter example. 
% 
% Copyright 2019 Friedemann Zenke, fzenke.net
%
% Based on examples by Dirk Eddelbuettel, Fran and others from 
% https://tex.stackexchange.com/questions/2317/latex-style-or-macro-for-detailed-response-to-referee-report
% 
% Licensed under cc by-sa 3.0 with attribution required.

\documentclass[11pt]{article}
\usepackage[utf8]{inputenc}
\usepackage{lipsum} % to generate some filler text
\usepackage{fullpage}
\usepackage{xcolor}
\usepackage{url}
\usepackage{multirow}
% import Eq and Section references from the main manuscript where needed
% \usepackage{xr}
% \externaldocument{manuscript}

% package needed for optional arguments
\usepackage{xifthen}
% define counters for reviewers and their points
\newcounter{reviewer}
\setcounter{reviewer}{0}
\newcounter{point}[reviewer]
\setcounter{point}{0}

% This refines the format of how the reviewer/point reference will appear.
\renewcommand{\thepoint}{C\,\arabic{point}} 

% command declarations for reviewer points and our responses
\newcommand{\reviewersection}{\stepcounter{reviewer} \bigskip \hrule
                  \section*{Reviewer \thereviewer}}

\newenvironment{point}
   {\refstepcounter{point} \bigskip \noindent {\textbf{Reviewer~Comment~\thepoint} } ---\ }
   {\par }

\newcommand{\shortpoint}[1]{\refstepcounter{point}  \bigskip \noindent 
	{\textbf{Reviewer~Comment~\thepoint} } ---~#1\par }

\newenvironment{reply}
   {\medskip \noindent \begin{sf}\textbf{Reply}:\  }
   {\medskip \end{sf}}

\newcommand{\shortreply}[2][]{\medskip \noindent \begin{sf}\textbf{Reply}:\  #2
	\ifthenelse{\equal{#1}{}}{}{ \hfill \footnotesize (#1)}%
	\medskip \end{sf}}

\begin{document}

\section*{Review of ``Estimated Covid-19 burden in Spain: ARCH underreported non-stationary time series''}
% General intro text goes here

% Let's start point-by-point with Reviewer 1
\reviewersection

% Point one description 
\begin{point}
Title, abstract, keywords weren't fitted together. The main approach ARCH wasn't interested within abstract in methods paragraph, while in title and keywords appear as main proposed approach.
	\label{pt:C1}
\end{point}

\begin{reply}
XXXXXXXXXXXXXXXXXXXXXX
\end{reply}

\begin{point}
The problems and objectives don't mentioned clearly in abstract.
	\label{pt:C2}
\end{point}

\begin{reply}
XXXXXXXXXXXXXXXXXXXXX

\textcolor{blue}{``XXXXXXXXXXXXXXXXX''}

\end{reply}

\begin{point}
``Estimation'' is not fitted with time series, while ``Forecasting'' or ``prediction'' is more appropriate.
	\label{pt:C3}
\end{point}

\begin{reply}
XXXXXXXX
\end{reply}

\begin{point}
The measurements of error should be used such as MAPE, RMSE, ans others.
	\label{pt:C4}
\end{point}

\begin{reply}
XXXXXXXXXXXXXXX
\end{reply}

\begin{point}
Discussions and conclusions should include more details about ARCH and its results.
	\label{pt:C5}
\end{point}

\begin{reply}
XXXXXXXXXXXXXXXXXXXXXX
\end{reply}

% Let's start point-by-point with Reviewer 2
\reviewersection

% Point one description 
\begin{point}
Good manuscript but the statistics in paper were beyond my expertise to comment. 
	\label{pt:C8}
\end{point}

\begin{reply}
We thank the reviewer for their critical reading of our work and their positive comment.
\end{reply}

% Let's start point-by-point with Reviewer 3
\reviewersection

% Point one description 
\begin{point}
In this paper, the authors employ a method to deal with underreported data and apply it to a dataset of COVID-19 cases in Spain. Although the method is of interest, the authors are not providing new methodological tools, but only applying existing ones. Besides, I have the following specific comments:
	\label{pt:C9}
\end{point}

\begin{reply}
We thank the reviewer for their critical reading of our work and their comments and suggestions, responded below.
\end{reply}

\begin{point}
The simulation study carried out by the authors is not motivated enough, in my opinion. The method has already been tested by their authors, so why performing this  simulation study?
	\label{pt:C10}
\end{point}

\begin{reply}
XXXXXXXXXXXXX
\end{reply}

\begin{point}
In Figures 1 and 2 I feel that something is wrong with the plot. For instance, in Figure 2, I don't understand the values around 2022-01. Maybe the upper estimation for that time point is greater than the maximum value fixed for the y-axis?
	\label{pt:C11}
\end{point}

\begin{reply}
XXXXXXXXXXXXX
\end{reply}

\begin{point}
I also feel that there might be some inconsistency between the results highlighted in page 18 and those provided in Table 2 (estimates of q for Aragon and Extremadura).
	\label{pt:C12}
\end{point}

\begin{reply}
XXXXXXXXXXXXX
\end{reply}

\begin{point}
In the Conclusions, there are some comments referred to the covariates included in the model. I would consider moving these results from the Supplementary Material to the main text.
	\label{pt:C13}
\end{point}

\begin{reply}
XXXXXXXXXXXXX
\end{reply}
\end{document}