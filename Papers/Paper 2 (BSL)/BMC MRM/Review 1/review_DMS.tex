% LaTeX rebuttal letter example. 
% 
% Copyright 2019 Friedemann Zenke, fzenke.net
%
% Based on examples by Dirk Eddelbuettel, Fran and others from 
% https://tex.stackexchange.com/questions/2317/latex-style-or-macro-for-detailed-response-to-referee-report
% 
% Licensed under cc by-sa 3.0 with attribution required.

\documentclass[11pt]{article}
\usepackage[utf8]{inputenc}
\usepackage{lipsum} % to generate some filler text
\usepackage{fullpage}
\usepackage{xcolor}
\usepackage{url}
\usepackage{multirow}
% import Eq and Section references from the main manuscript where needed
% \usepackage{xr}
% \externaldocument{manuscript}

% package needed for optional arguments
\usepackage{xifthen}
% define counters for reviewers and their points
\newcounter{reviewer}
\setcounter{reviewer}{0}
\newcounter{point}[reviewer]
\setcounter{point}{0}

% This refines the format of how the reviewer/point reference will appear.
\renewcommand{\thepoint}{C\,\arabic{point}} 

% command declarations for reviewer points and our responses
\newcommand{\reviewersection}{\stepcounter{reviewer} \bigskip \hrule
                  \section*{Reviewer \thereviewer}}

\newenvironment{point}
   {\refstepcounter{point} \bigskip \noindent {\textbf{Reviewer~Comment~\thepoint} } ---\ }
   {\par }

\newcommand{\shortpoint}[1]{\refstepcounter{point}  \bigskip \noindent 
	{\textbf{Reviewer~Comment~\thepoint} } ---~#1\par }

\newenvironment{reply}
   {\medskip \noindent \begin{sf}\textbf{Reply}:\  }
   {\medskip \end{sf}}

\newcommand{\shortreply}[2][]{\medskip \noindent \begin{sf}\textbf{Reply}:\  #2
	\ifthenelse{\equal{#1}{}}{}{ \hfill \footnotesize (#1)}%
	\medskip \end{sf}}

\begin{document}

\section*{Review of ``Estimated Covid-19 burden in Spain: ARCH underreported non-stationary time series''}
% General intro text goes here

% Let's start point-by-point with Reviewer 1
\reviewersection

We thank the reviewer for their critical reading of our work and their comments and suggestions, responded below.

% Point one description 
\begin{point}
Title, abstract, keywords weren't fitted together. The main approach ARCH wasn't interested within abstract in methods paragraph, while in title and keywords appear as main proposed approach.
	\label{pt:C1}
\end{point}

\begin{reply}
We agree with the reviewer. The \textit{Methods} paragraph in the abstract has been updated to:

\textcolor{blue}{``\textit{\textbf{Methods:}} In this work, we explore the performance of Bayesian Synthetic Likelihood to estimate the parameters of a model based on AutoRegressive Conditional Heteroskedastic time series capable of dealing with misreported information and to reconstruct the most likely evolution of the phenomenon. The performance of the proposed methodology is evaluated through a comprehensive simulation study and illustrated by reconstructing the weekly Covid-19 incidence in each Spanish Autonomous Community.''}

The \textit{keywords} have also been updated, replacing GARCH to ARCH, which is the main proposed approach as pointed out by the reviewer.
\end{reply}

\begin{point}
The problems and objectives don't mentioned clearly in abstract.
	\label{pt:C2}
\end{point}

\begin{reply}
XXXXXXXXXXXXXXXXXXXXX

\textcolor{blue}{``XXXXXXXXXXXXXXXXX''}

\end{reply}

\begin{point}
``Estimation'' is not fitted with time series, while ``Forecasting'' or ``prediction'' is more appropriate.
	\label{pt:C3}
\end{point}

\begin{reply}
XXXXXXXX
\end{reply}

\begin{point}
The measurements of error should be used such as MAPE, RMSE, ans others.
	\label{pt:C4}
\end{point}

\begin{reply}
XXXXXXXXXXXXXXX
\end{reply}

\begin{point}
Discussions and conclusions should include more details about ARCH and its results.
	\label{pt:C5}
\end{point}

\begin{reply}
XXXXXXXXXXXXXXXXXXXXXX
\end{reply}

% Let's start point-by-point with Reviewer 2
\reviewersection

% Point one description 
\begin{point}
Good manuscript but the statistics in paper were beyond my expertise to comment. 
	\label{pt:C8}
\end{point}

\begin{reply}
We thank the reviewer for their critical reading of our work and their positive comment.
\end{reply}

% Let's start point-by-point with Reviewer 3
\reviewersection

% Point one description 
\begin{point}
In this paper, the authors employ a method to deal with underreported data and apply it to a dataset of COVID-19 cases in Spain. Although the method is of interest, the authors are not providing new methodological tools, but only applying existing ones. Besides, I have the following specific comments:
	\label{pt:C9}
\end{point}

\begin{reply}
We thank the reviewer for their critical reading of our work and their comments and suggestions, responded below.
\end{reply}

\begin{point}
The simulation study carried out by the authors is not motivated enough, in my opinion. The method has already been tested by their authors, so why performing this  simulation study?
	\label{pt:C10}
\end{point}

\begin{reply}
The method used to estimate the parameters of the model (Bayesian synthetic likelihood) is known and well-tested, but to the best of our knowledge it has never been used in the context of AutoRegressive Conditional Heteroskedasticity (ARCH) underreported models, and this is why its performance is assessed through the simulation study. To clarify, the following paragraph has been added to the manuscript (page XXX):

\textcolor{blue}{``''}

\end{reply}

\begin{point}
In Figures 1 and 2 I feel that something is wrong with the plot. For instance, in Figure 2, I don't understand the values around 2022-01. Maybe the upper estimation for that time point is greater than the maximum value fixed for the y-axis?
	\label{pt:C11}
\end{point}

\begin{reply}
As the reviewer points out, there is something curious about the values around 2022-01. The interpretation is that in all simulations this particular value is detected as not being underreported ($Y_{2022-01} = X_{2022-01}$). This is why the estimated and registered values and the 95\% credible interval for this value are all equal, as represented in Figure 1 and Figure 2. To clarify, the following sentences has been added to the manuscript (page XXX):

\textcolor{blue}{``''}

\end{reply}

\begin{point}
I also feel that there might be some inconsistency between the results highlighted in page 18 and those provided in Table 2 (estimates of q for Aragon and Extremadura).
	\label{pt:C12}
\end{point}

\begin{reply}
We thank the reviewer for noticing these inconsistencies between the estimates provided in the text and in Table 2. Those provided in Table 2 were correct, the values discussed in the text have been corrected accordingly.
\end{reply}

\begin{point}
In the Conclusions, there are some comments referred to the covariates included in the model. I would consider moving these results from the Supplementary Material to the main text.
	\label{pt:C13}
\end{point}

\begin{reply}
XXXXXXXXXXXXX
\end{reply}
\end{document}