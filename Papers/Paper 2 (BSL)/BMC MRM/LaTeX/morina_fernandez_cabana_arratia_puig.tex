%% BioMed_Central_Tex_Template_v1.06
%%                                      %
%  bmc_article.tex            ver: 1.06 %
%                                       %

%%IMPORTANT: do not delete the first line of this template
%%It must be present to enable the BMC Submission system to
%%recognise this template!!

%%%%%%%%%%%%%%%%%%%%%%%%%%%%%%%%%%%%%%%%%
%%                                     %%
%%  LaTeX template for BioMed Central  %%
%%     journal article submissions     %%
%%                                     %%
%%          <8 June 2012>              %%
%%                                     %%
%%                                     %%
%%%%%%%%%%%%%%%%%%%%%%%%%%%%%%%%%%%%%%%%%


%%%%%%%%%%%%%%%%%%%%%%%%%%%%%%%%%%%%%%%%%%%%%%%%%%%%%%%%%%%%%%%%%%%%%
%%                                                                 %%
%% For instructions on how to fill out this Tex template           %%
%% document please refer to Readme.html and the instructions for   %%
%% authors page on the biomed central website                      %%
%% http://www.biomedcentral.com/info/authors/                      %%
%%                                                                 %%
%% Please do not use \input{...} to include other tex files.       %%
%% Submit your LaTeX manuscript as one .tex document.              %%
%%                                                                 %%
%% All additional figures and files should be attached             %%
%% separately and not embedded in the \TeX\ document itself.       %%
%%                                                                 %%
%% BioMed Central currently use the MikTex distribution of         %%
%% TeX for Windows) of TeX and LaTeX.  This is available from      %%
%% http://www.miktex.org                                           %%
%%                                                                 %%
%%%%%%%%%%%%%%%%%%%%%%%%%%%%%%%%%%%%%%%%%%%%%%%%%%%%%%%%%%%%%%%%%%%%%

%%% additional documentclass options:
%  [doublespacing]
%  [linenumbers]   - put the line numbers on margins

%%% loading packages, author definitions

%\documentclass[twocolumn]{bmcart}% uncomment this for twocolumn layout and comment line below
\documentclass{bmcart}

%%% Load packages
\usepackage{amsthm,amsmath}
%\RequirePackage{natbib}
%\RequirePackage[authoryear]{natbib}% uncomment this for author-year bibliography
\RequirePackage{hyperref}
\usepackage[utf8]{inputenc} %unicode support
%\usepackage[applemac]{inputenc} %applemac support if unicode package fails
%\usepackage[latin1]{inputenc} %UNIX support if unicode package fails
\usepackage{multirow}
\usepackage{amssymb}

%%%%%%%%%%%%%%%%%%%%%%%%%%%%%%%%%%%%%%%%%%%%%%%%%
%%                                             %%
%%  If you wish to display your graphics for   %%
%%  your own use using includegraphic or       %%
%%  includegraphics, then comment out the      %%
%%  following two lines of code.               %%
%%  NB: These line *must* be included when     %%
%%  submitting to BMC.                         %%
%%  All figure files must be submitted as      %%
%%  separate graphics through the BMC          %%
%%  submission process, not included in the    %%
%%  submitted article.                         %%
%%                                             %%
%%%%%%%%%%%%%%%%%%%%%%%%%%%%%%%%%%%%%%%%%%%%%%%%%


\def\includegraphic{}
\def\includegraphics{}



%%% Put your definitions there:
\startlocaldefs
\endlocaldefs


%%% Begin ...
\begin{document}

%%% Start of article front matter
\begin{frontmatter}

\begin{fmbox}
\dochead{Research}

%%%%%%%%%%%%%%%%%%%%%%%%%%%%%%%%%%%%%%%%%%%%%%
%%                                          %%
%% Enter the title of your article here     %%
%%                                          %%
%%%%%%%%%%%%%%%%%%%%%%%%%%%%%%%%%%%%%%%%%%%%%%

\title{Estimated Covid-19 burden in Spain: ARCH underreported non-stationary time series}

\author[
   addressref={aff1},                   % id's of addresses, e.g. {aff1,aff2}
   corref={aff1},                       % id of corresponding address, if any                       % id's of article notes, if any
   email={dmorina@ub.edu}   % email address
]{\inits{DM}\fnm{David} \snm{Mori\~na}}
\author[
   addressref={aff2},
   email={amanda@mat.uab.cat}
]{\inits{AF-F}\fnm{Amanda} \snm{Fern\'andez-Fontelo}}
\author[
   addressref={aff2},
   email={acabana@mat.uab.cat}
]{\inits{AC}\fnm{Alejandra} \snm{Caba\~na}}
\author[
   addressref={aff3},
   email={argimiro@cs.upc.edu}
]{\inits{AA}\fnm{Argimiro} \snm{Arratia}}
\author[
   addressref={aff2,aff4},
   email={ppuig@mat.uab.cat}
]{\inits{PP}\fnm{Pedro} \snm{Puig}}

%%%%%%%%%%%%%%%%%%%%%%%%%%%%%%%%%%%%%%%%%%%%%%
%%                                          %%
%% Enter the authors' addresses here        %%
%%                                          %%
%% Repeat \address commands as much as      %%
%% required.                                %%
%%                                          %%
%%%%%%%%%%%%%%%%%%%%%%%%%%%%%%%%%%%%%%%%%%%%%%

\address[id=aff1]{%
  \orgname{Department of Econometrics, Statistics and Applied Economics, Riskcenter-IREA, Universitat de Barcelona (UB)},
  \street{Avinguda Diagonal, 690},
  \postcode{08034}
  \city{Barcelona},
  \cny{Spain}
}
\address[id=aff2]{%                           % unique id
  \orgname{Departament de Matem\`atiques, Universitat Aut\`onoma de Barcelona (UAB)}, % university, etc
  \street{Edifici C, Campus de Bellaterra},                     %
  \postcode{08193}                                % post or zip code
  \city{Cerdanyola del Vall\`es},                              % city
  \cny{Spain}                                    % country
}
\address[id=aff3]{%
  \orgname{Department of Computer Science, Universitat Polit\`ecnica de Catalunya (UPC)},
  %\street{Avinguda Diagonal, 690},
  %\postcode{08034}
  \city{Barcelona},
  \cny{Spain}
}
\address[id=aff4]{%                           % unique id
  \orgname{Centre de Recerca Matem\`atica}, % university, etc
  \street{Universitat Aut\`onoma de Barcelona (UAB), Edifici C, Campus de Bellaterra},                     %
  \postcode{08193}                                % post or zip code
  \city{Cerdanyola del Vall\`es},                              % city
  \cny{Spain}                                    % country
}

%%%%%%%%%%%%%%%%%%%%%%%%%%%%%%%%%%%%%%%%%%%%%%
%%                                          %%
%% Enter short notes here                   %%
%%                                          %%
%% Short notes will be after addresses      %%
%% on first page.                           %%
%%                                          %%
%%%%%%%%%%%%%%%%%%%%%%%%%%%%%%%%%%%%%%%%%%%%%%

%\begin{artnotes}
%\note{Sample of title note}     % note to the article
%\note[id=n1]{Equal contributor} % note, connected to author
%\end{artnotes}

\end{fmbox}% comment this for two column layout

%%%%%%%%%%%%%%%%%%%%%%%%%%%%%%%%%%%%%%%%%%%%%%
%%                                          %%
%% The Abstract begins here                 %%
%%                                          %%
%% Please refer to the Instructions for     %%
%% authors on http://www.biomedcentral.com  %%
%% and include the section headings         %%
%% accordingly for your article type.       %%
%%                                          %%
%%%%%%%%%%%%%%%%%%%%%%%%%%%%%%%%%%%%%%%%%%%%%%

\begin{abstractbox}

\begin{abstract} % abstract
\parttitle{Background} The problem of dealing with misreported data is very common in a wide range of contexts for different reasons. The current situation caused by the Covid-19 worldwide pandemic is a clear example, where the data provided by official sources were not always reliable due to data collection issues and to the high proportion of asymptomatic cases. \parttitle{Methods} In this work, we explore the performance of Bayesian Synthetic Likelihood to estimate the parameters of a model capable of dealing with misreported information and to reconstruct the most likely evolution of the phenomenon. The performance of the proposed methodology is evaluated through a comprehensive simulation study and illustrated by reconstructing the weekly Covid-19 incidence in each Spanish Autonomous Community. \parttitle{Results} Only around 60\% of the Covid-19 cases in the period 2020/02/23-2022/02/27 were reported in Spain, showing significant differences in the severity of underreporting across the regions. \parttitle{Conclusions} The proposed methodology provides public health decision-makers with a valuable tool in order to improve the assessment of a disease evolution under different scenarios.
\end{abstract}

\begin{keyword}
  \kwd{continuous time series}
  \kwd{mixture distributions}
  \kwd{under-reported data}
  \kwd{GARCH models}
  \kwd{infectious diseases}
  \kwd{Covid-19}
  \kwd{Bayesian synthetic likelihood}
\end{keyword}

% MSC classifications codes, if any
%\begin{keyword}[class=AMS]
%\kwd[Primary ]{}
%\kwd{}
%\kwd[; secondary ]{}
%\end{keyword}

\end{abstractbox}
%
%\end{fmbox}% uncomment this for twcolumn layout

\end{frontmatter}

%%%%%%%%%%%%%%%%%%%%%%%%%%%%%%%%%%%%%%%%%%%%%%
%%                                          %%
%% The Main Body begins here                %%
%%                                          %%
%% Please refer to the instructions for     %%
%% authors on:                              %%
%% http://www.biomedcentral.com/info/authors%%
%% and include the section headings         %%
%% accordingly for your article type.       %%
%%                                          %%
%% See the Results and Discussion section   %%
%% for details on how to create sub-sections%%
%%                                          %%
%% use \cite{...} to cite references        %%
%%  \cite{koon} and                         %%
%%  \cite{oreg,khar,zvai,xjon,schn,pond}    %%
%%  \nocite{smith,marg,hunn,advi,koha,mouse}%%
%%                                          %%
%%%%%%%%%%%%%%%%%%%%%%%%%%%%%%%%%%%%%%%%%%%%%%

%%%%%%%%%%%%%%%%%%%%%%%%% start of article main body
% <put your article body there>

%%%%%%%%%%%%%%%%
%% Background %%
%%
\section*{Background}\label{intro}
The Covid-19 pandemic that is hitting the world since late 2019 has made evident that having quality data is essential in the decision making chain, especially in epidemiology but also in many other fields. There is an enormous global concern around this disease, leading the World Health Organization (WHO) to declare public health emergency \cite{Sohrabi2020}. Many methodological efforts have been made to deal with misreported Covid-19 data, following ideas introduced in the literature since the late nineties \cite{Bernard2014,Arendt2013,Rosenman2006,Alfonso2015,Winkelmann1996,Gibbons2014}. These proposals range from the usage of multiplication factors \cite{Stocks2018} to Markov-based models \cite{Azmon2014,Magal2018} or spatio-temporal models \cite{Stoner2019}. Additionally, a new R \cite{RCoreTeam2019} package able to fitting endemic-epidemic models based on approximative maximum likelihood to underreported count data has been recently published \cite{JohannesBracher2019}. However, as a large proportion of the cases run asymptomatically \cite{Oran2020} and mild symptoms could have been easily confused with those of similar diseases at the beginning of the pandemic, its reasonable to expect that Covid-19 incidence has been notably underreported. Very recently several approaches based on discrete time series have been proposed \cite{Fernandez-Fontelo2016,FernandezFontelo2019,Fernandez-Fontelo2020} although there is a lack of continuous time series models capable of dealing with misreporting, a characteristic of the Covid-19 data and typically present in infectious diseases modeling. In this sense, a new approach for longitudinal data not accounting for temporal correlations is introduced in \cite{Morina2021} and a model capable of dealing with temporal structures using a different approach is presented in \cite{Morina2020}. A typical limitation of these kinds of models is the computational effort needed in order to properly estimate the parameters.

Synthetic likelihood is a recent and very powerful alternative for parameter estimation in a simulation based schema when the likelihood is intractable and, conversely, the generation of new observations given the values of the parameters is feasible. The method was introduced in \cite{Wood2010} and placed into a Bayesian framework in \cite{Price2018}, showing that it could be scaled to high dimensional problems and can be adapted in an easier way than other alternatives like approximate Bayesian computation (ABC). The method takes a vector summary statistic informative about the parameters and assumes it is multivariate normal, estimating the unknown mean and covariance matrix by simulation to obtain an approximate likelihood function of the multivariate normal. 

\section*{Methods}\label{methods}
AutoRegressive Conditional Heteroskedasticity (ARCH) models are a well-known approach to fitting time series data where the variance error is believed to be serially correlated. Consider an unobservable process $X_t$ following an AutoRegressive ($AR(1)$) model with ARCH(1) errors structure, defined by
$$
X_t = \phi_0 + \phi_1 \cdot X_{t-1} + Z_t,
$$

where $Z^2_t=\alpha_0+\alpha_1 \cdot Z^2_{t-1} + \epsilon_t,$ being $\epsilon_t \sim N(\mu_{\epsilon}(t), \sigma_{\epsilon}^2)$. The process $X_t$ represents the actual Covid-19 incidence. In our setting, this process $X_t$ cannot be directly observed, and all we can see is a part of it, expressed as

\begin{equation}\label{morina:eq1}
    Y_t=\left\{
                \begin{array}{ll}
                  X_t \text{ with probability } 1-\omega \\
                  q \cdot X_t \text{ with probability } \omega,
                \end{array}
              \right.
\end{equation}

where $q$ is the overall intensity of misreporting (if $0 < q < 1$ the observed process $Y_t$ would be underreported while if $q > 1$ the observed process $Y_t$ would be overreported) and $\omega$ can be interpreted as the overall frequency of misreporting (proportion of misreported observations). To model consistently the spread of the disease, the expectation of the innovations $\epsilon_t$ is linked to a simplified version of the well-known compartimental Susceptible-Infected-Recovered (SIR) model. At any time $t \in \mathbb{R}$ there are three kinds of individuals: Healthy individuals susceptible to be infected ($S(t)$), infected individuals who are transmitting the disease at a certain speed ($I(t)$) and individuals who have suffered the disease, recovered and cannot be infected again ($R(t)$). As shown in \cite{Fernandez-Fontelo2020}, the number of affected individuals at time $t$, $A(t) = I(t) + R(t)$ can be approximated by

\begin{equation}\label{eq:SIR}
  A(t) = \frac{M^{*}(\beta_0, \beta_1, \beta_2, t) A_0 e^{kt}}{M^{*}(\beta_0, \beta_1, \beta_2, t)+A_0(e^{kt}-1)},
\end{equation}
where $M^{*}(\beta_0, \beta_1, \beta_2, t) = \beta_0+\beta_1 \cdot C_1(t) + \beta_2 \cdot C_2(t)$, being $C_1(t)$ and $C_2(t)$ dummy variables indicating if time $t$ corresponds to a period where a mandatory confinment was implemented by the government and if the number of people with at least one dose of a Covid-19 vaccine in Spain was over 50\% respectively. At any time $t$ the condition $S(t) + I(t) + R(t) = N$ is fulfilled. The expression~(\ref{eq:SIR}) allow us to incorporate the behaviour of the epidemics in a realistic way, defining $\mu_{\epsilon}(t) = A(t) - A(t-1)$, the new affected cases produced at time $t$.

The Bayesian Synthetic Likelihood (BSL) simulations are based on the described model and the chosen summary statistics are the mean, standard deviation and the three first coefficients of autocorrelation of the observed process. Parameter estimation was carried out by means of the \textit{BSL} \cite{BSLManual,An2019} package for R \cite{RCoreTeam2019}. Taking into account the posterior distribution of the estimated parameters, the most likely unobserved process is reconstructed, resulting in a probability distribution at each time point. The prior of each parameter is set to be uniform on the corresponding feasible region of the parameter space and zero elsewhere.

\section*{Results}\label{results}
This section presents the results of the analyses using the proposed methodology over a real data set and they are compared to the most common alternatives. The performance of the method is also studied by means of a comprehensive simulation study, with and without covariates.

The performance and an application of the proposed methodology are studied through a comprehensive simulation study and a real dataset on Covid-19 incidence in Spain on this Section.

\subsection*{Simulation study}\label{sim}
A thorough simulation study has been conducted to ensure that the model behaves as expected, including $ARCH(1)$, $AR(1)$, $MA(1)$ and $ARMA(1, 1)$ structures for the hidden process $X_t$ defined as

\begin{equation}\begin{array}{c}
  X_t = \phi_0 + \phi_1 \cdot X_{t-1} + Z_t, Z^2_t=\alpha_0+\alpha_1 \cdot Z^2_{t-1} + \epsilon_t \text{ (ARCH(1))} \\
  X_t = \phi_0 + \alpha \cdot X_{t-1} + \epsilon_t \text{ (AR(1))} \\
  X_t = \phi_0 + \theta \cdot \epsilon_{t-1} + \epsilon_t \text{ (MA(1))} \\
  X_t = \phi_0 + \alpha \cdot X_{t-1} + \theta \cdot \epsilon_{t-1} + \epsilon_t \text{ (ARMA(1, 1))}
\end{array}\end{equation}
where $\epsilon_t \sim N(\mu_{\epsilon}(t), \sigma_{\epsilon}^2)$.

The values for the parameters $\phi_1$, $\alpha_0$, $\alpha_1$, $\alpha$, $\theta$, $q$ and $\omega$ ranged from 0.1 to 0.9 for each parameter. Average absolute bias, average interval length (AIL) and average 95\% credible interval coverage are shown in Table~\ref{tab:estim_sim}. To summarise model robustness, these values are averaged over all combinations of parameters, considering their prior distribution is a Dirac's delta with all probability concentrated in the corresponding parameter value. 

For each autocorrelation structure and parameters combination, a random sample of size $n = 1000$ has been generated using the R function \textit{arima.sim}, and the parameters $m=log(M^*)$ and $\beta$ have been fixed to $0.2$ and $0.4$ respectively. Several values for these parameters were considered but no substantial differences in the model performance were observed related to the value of these parameters or sample size, besides a poorer coverage for lower sample sizes, as could be expected.

%Table 1 should be placed around here.

\subsection*{Real incidence of Covid-19 in Spain}\label{covid}
The betacoronavirus SARS-CoV-2 has been identified as the causative agent of an unprecedented world-wide outbreak of pneumonia starting in December 2019 in the city of Wuhan (China) \cite{Sohrabi2020}, named as Covid-19. Considering that many cases run without developing symptoms or just with very mild symptoms, it is reasonable to assume that the incidence of this disease has been underregistered. This work focuses on the weekly Covid-19 incidence registered in Spain in the period (2020/02/23-2022/02/27). It can be seen in Figure~\ref{morina:fig1} that the registered data (turquoise) reflect only a fraction of the actual incidence (red). The grey area corresponds to 95\% probability of the posterior distribution of the weekly number of new cases (the lower and upper limits of this area represent the percentile 2.5\% and 97.5\% respectively), and the dotted red line corresponds to its median.

%Figure 1 should be placed around here.

In the considered period, the official sources reported 11,056,797 Covid-19 cases in Spain, while the model estimates a total of 25,283,406 cases (only 43.73\% of actual cases were reported). This work also revealed that while the frequency of underreporting is extremely high for all regions (values of $\hat{\omega}$ over 0.90 in all cases), the intensity of this underreporting is not uniform across the considered regions: Arag\'on is the CCAA with highest underreporting intensity ($\hat{q}=0.05$) while Extremadura is the region where the estimated values are closest to the number of reported cases ($\hat{q}=0.50$). Detailed underreported parameter estimates for each region can be found in Table~\ref{tab:ccaa_est}. Although the main impact of the vaccination programmes can be seen in mortality data, the results of this work also showed a significant decrease in the weekly number of cases as well in all CCAA except Arag\'on. Figure~\ref{morina:fig2} represents the estimated and registered Covid-19 weekly incidence globally for Spain. The estimated impact of the considered covariates is available in Appendix A (Table~\ref{tab:ccaa_est2}).

%Table 2 should be placed around here.

Figure~\ref{morina:fig2} shows the evolution of the registered (turquoise) and estimated (red) weekly number of Covid-19 cases in Spain in the period 2020/02/23-2022/02/27.

%Figure 2 should be placed around here.

The global differences in underreporting magnitudes across regions can be represented by the percentage of reported cases in each CCAA (compared to model estimates), as shown in Figure~\ref{morina:fig3}.

%Figure 3 should be placed around here.

\section*{Discussion}\label{discussion}
Although it is very common in biomedical and epidemiological research to get data from disease registries, there is a concern about their reliability, and there have been some recent efforts to standardize the protocols in order to improve the accuracy of health information registries (see for instance \cite{Kodra2018,Harkener2019}). However, as the Covid-19 pandemic situation has made evident, it is not always possible to implement these recommendations in a proper way. 

Another work analyzing the cumulated burden of Covid-19 in Spain (\cite{morina_cumulated_2021}) estimated that only around 21\% of the cases were reported in the period 2020/01/01-2020/06/01, but it should be taken into account that it seems reasonable to assume that the underreporting intensity was higher at the early stages of the pandemic, and therefore a lower overall underreporing is expected in the longer period considered in the present study. Additionally, the presented methodology allows for a real time monitoring and not only cumulated over a time period.

Having accurate data is key in order to provide public health decision-makers with reliable information, which can also be used to improve the accuracy of dynamic models aimed to estimate the spread of the disease \cite{Zhao2020} and to predict its behavior. 

\section*{Conclusions}\label{conclusions}
The proposed methodology can deal with misreported (over- or under-reported) data in a very natural and straightforward way, and is able to reconstruct the most likely hidden process, providing public health decision-makers with a valuable tool in order to predict the evolution of the disease under different scenarios.

The analysis of the Spanish Covid-19 data shows that in average only around 60\% of the cases in the period 2020/02/23-2022/02/27 were reported, and that there are significant differences in the severity of underreporting across the regions. The impact of the vaccination programme can also be assessed, achieving a significant decrease in the Covid-19 incidence in almost all regions after 50\% of the population had one dose at least (although these results would probably be notably different if including SARS-CoV-2 immunity-escape variants like BA.4 or BA.5, which are currently predominant in many countries), while the impact of the mandatory lockdown could only be detected by the model in 7 regions.

The simulation study shows that the proposed methodology behaves as expected and that the parameters used in the simulations, under different autocorrelation structures, can be recovered, even with severely underreported data.

\section*{Abbreviations}
\begin{itemize}
 \item ABC: Approximate Bayesian computation.
 \item AIL: Average interval length.
 \item AR: AutoRegressive model.
 \item ARCH: AutoRegressive Conditional Heteroskedasticity model.
 \item ARMA: AutoRegressive Moving Average model.
 \item BSL: Bayesian Synthetic Likelihood.
 \item CCAA: Spanish autonomous community.
 \item CrI: Credible interval.
 \item EM: Expectation-Maximization.
 \item MA: Moving average model.
 \item SIR: Susceptible-Infected-Recovered model.
\end{itemize}

%%%%%%%%%%%%%%%%%%%%%%%%%%%%%%%%%%%%%%%%%%%%%%
%%                                          %%
%% Backmatter begins here                   %%
%%                                          %%
%%%%%%%%%%%%%%%%%%%%%%%%%%%%%%%%%%%%%%%%%%%%%%

\section*{Declarations}
\begin{backmatter}
\section*{Ethics approval and consent to participate}
Not applicable.

\section*{Consent for publication}
Not applicable.

\section*{Availability of data and materials}
Data and R codes used in the described analyses are available in the GitHub repository \url{https://github.com/dmorinya/BSLCovidSpain}.

\section*{Competing interests}
The authors declare no competing interests.

\section*{Funding}
Research funded by Fundaci\'on MAPFRE. This work was partially supported by grant RTI2018-096072-B-I00 from the Spanish Ministry of Science and Innovation and by the Spanish State Research Agency, through the Severo Ochoa and María de Maeztu Program for Centers and Units of Excellence in R\&D (CEX2020–001084-M). A.F-F acknowledges Agencia Estatal de Investigaci\'on for the financial support IJC2020-045188I/AEI/10.13039/501100011033.

\section*{Author's contributions}
DM, AF-F, AC, AA and PP participated in the development of the statistical model. DM and PP derived the described properties, and DM implemented the model in R software and conducted the analyses. All authors have read and approved the manuscript.

\section*{Acknowledgements}
Not applicable.

%%%%%%%%%%%%%%%%%%%%%%%%%%%%%%%%%%%%%%%%%%%%%%%%%%%%%%%%%%%%%
%%                  The Bibliography                       %%
%%                                                         %%
%%  Bmc_mathpys.bst  will be used to                       %%
%%  create a .BBL file for submission.                     %%
%%  After submission of the .TEX file,                     %%
%%  you will be prompted to submit your .BBL file.         %%
%%                                                         %%
%%                                                         %%
%%  Note that the displayed Bibliography will not          %%
%%  necessarily be rendered by Latex exactly as specified  %%
%%  in the online Instructions for Authors.                %%
%%                                                         %%
%%%%%%%%%%%%%%%%%%%%%%%%%%%%%%%%%%%%%%%%%%%%%%%%%%%%%%%%%%%%%

% if your bibliography is in bibtex format, use those commands:
\bibliographystyle{bmc-mathphys}
\bibliography{morina_fernandez_cabana_arratia_puig}      % Bibliography file (usually '*.bib' )

\section*{Appendix A}
The impact of covariates for each Spanish region is reported in Table~\ref{tab:ccaa_est2}.

% Table 3 should be placed around here.

%%%%%%%%%%%%%%%%%%%%%%%%%%%%%%%%%%%
%%                               %%
%% Figures                       %%
%%                               %%
%% NB: this is for captions and  %%
%% Titles. All graphics must be  %%
%% submitted separately and NOT  %%
%% included in the Tex document  %%
%%                               %%
%%%%%%%%%%%%%%%%%%%%%%%%%%%%%%%%%%%

%%
%% Do not use \listoffigures as most will included as separate files

\section*{Figures}

\begin{figure}[ht]
  \caption{\label{morina:fig1} Registered and estimated weekly new Covid-19 cases in each Spanish region.}
\end{figure}

\begin{figure}[ht]
  \caption{\label{morina:fig2} Registered and estimated weekly new Covid-19 cases globally in Spain.}
\end{figure}

\begin{figure}[ht]
  \caption{\label{morina:fig3} Percentage of reported cases in each CCAA.}
\end{figure}

%%%%%%%%%%%%%%%%%%%%%%%%%%%%%%%%%%%
%%                               %%
%% Tables                        %%
%%                               %%
%%%%%%%%%%%%%%%%%%%%%%%%%%%%%%%%%%%

%% Use of \listoftables is discouraged.
%%
\section*{Tables}

\begin{table}[h!]
\tiny\sf\centering
\caption{Model performance measures (average absolute bias, average interval length and average coverage) summary based on a simulation study.\label{tab:estim_sim}}
\begin{tabular}{ccccc}
\hline
Structure & Parameter & Bias & AIL & Coverage (\%)\\
\hline
\multirow{9}{*}{$ARCH(1)$}    & $\hat{\phi_0}$            &-0.377 & 3.586 & 68.77\% \\
                              & $\hat{\phi_1}$            & 0.122 & 0.525 & 66.08\% \\
                              & $\hat{\alpha_0}$          &-0.296 & 1.351 & 74.72\% \\
                              & $\hat{\alpha_1}$          &-0.085 & 0.920 & 77.34\% \\
                              & $\hat{\omega}$            &-0.020 & 0.234 & 83.71\% \\
                              & $\hat{q}$                 &-0.022 & 0.167 & 85.06\% \\
                              & $\hat{m}$                 &-0.226 & 0.783 & 79.17\% \\
                              & $\hat{\beta}$             &-0.734 & 3.581 & 76.83\% \\
                              & $\hat{\sigma_{\epsilon}}$ &-1.540 & 3.323 & 63.65\% \\
\hline
\multirow{7}{*}{$AR(1)$}      & $\hat{\phi_0}$            &-0.983 & 5.189 & 70.10\% \\
                              & $\hat{\alpha}$            & 0.043 & 0.814 & 92.46\% \\
                              & $\hat{\omega}$            &-0.003 & 0.111 & 94.10\% \\
                              & $\hat{q}$                 &-0.001 & 0.014 & 89.03\% \\
                              & $\hat{m}$                 & 0.001 & 0.190 & 75.17\% \\
                              & $\hat{\beta}$             & 0.007 & 0.192 & 74.49\% \\
                              & $\hat{\sigma_{\epsilon}}$ &-1.689 & 4.718 & 81.07\% \\
\hline
\multirow{7}{*}{$MA(1)$}      & $\hat{\phi_0}$            &-1.241 & 5.171 & 68.31\% \\
                              & $\hat{\theta}$            & 0.051 & 0.818 & 90.40\% \\
                              & $\hat{\omega}$            &-0.005 & 0.108 & 95.06\% \\
                              & $\hat{q}$                 &-0.001 & 0.014 & 87.24\% \\
                              & $\hat{m}$                 &-0.002 & 0.187 & 76.95\% \\
                              & $\hat{\beta}$             & 0.004 & 0.190 & 80.38\% \\
                              & $\hat{\sigma_{\epsilon}}$ &-1.619 & 4.679 & 83.95\% \\
\hline
\multirow{8}{*}{$ARMA(1, 1)$} & $\hat{\phi_0}$            & -1.834 & 5.107 & 61.01\% \\
                              & $\hat{\alpha}$            & 0.062  & 0.799 & 89.39\% \\
                              & $\hat{\theta}$            & 0.011  & 0.873 & 96.86\% \\
                              & $\hat{\omega}$            & -0.001 & 0.014 & 88.32\% \\
                              & $\hat{q}$                 & -0.005 & 0.109 & 94.97\% \\
                              & $\hat{m}$                 & 0.002  & 0.184 & 78.49\% \\
                              & $\hat{\beta}$             & 0.004  & 0.183 & 78.01\% \\
                              & $\hat{\sigma_{\epsilon}}$ & -1.828 & 4.631 & 74.74\% \\
\hline
\end{tabular}
\end{table}

\begin{table}[h]
\small\sf\centering
\caption{Estimated underreported frequency and intensity for each Spanish CCAA. Reported values correspond to the median and percentiles 2.5\% and 97.5\% of the corresponding posterior distribution.\label{tab:ccaa_est}}
\begin{tabular}{ccc}
\hline
CCAA & Parameter & Estimate (95\% CrI)\\
\hline
\multirow{2}{*}{Andaluc\'ia}  & $\hat{\omega}$  & 0.97 (0.95 - 0.99) \\
                              & $\hat{q}$       & 0.44 (0.41 - 0.48) \\
\hline
\multirow{2}{*}{Arag\'on}    & $\hat{\omega}$  & 0.98 (0.97 - 0.99) \\
                             & $\hat{q}$       & 0.28 (0.27 - 0.32) \\
\hline
\multirow{2}{*}{Asturies}    & $\hat{\omega}$  & 0.97 (0.90 - 0.99) \\
                             & $\hat{q}$       & 0.40 (0.37 - 0.53) \\
\hline
\multirow{2}{*}{Cantabria}    & $\hat{\omega}$  & 0.97 (0.95 - 0.99) \\
                              & $\hat{q}$       & 0.30 (0.28 - 0.35) \\
\hline
\multirow{2}{*}{Castilla y Le\'on}    & $\hat{\omega}$  & 0.98 (0.95 - 0.99) \\
                                      & $\hat{q}$       & 0.36 (0.32 - 0.41) \\
\hline
\multirow{2}{*}{Castilla - La Mancha}    & $\hat{\omega}$  & 0.98 (0.96 - 0.99) \\
                                         & $\hat{q}$       & 0.33 (0.31 - 0.36) \\
\hline
\multirow{2}{*}{Canarias}    & $\hat{\omega}$  & 0.98 (0.96 - 0.99) \\
                             & $\hat{q}$       & 0.35 (0.32 - 0.38) \\
\hline
\multirow{2}{*}{Catalunya}    & $\hat{\omega}$  & 0.98 (0.96 - 0.99) \\
                              & $\hat{q}$       & 0.30 (0.27 - 0.34) \\
\hline
\multirow{2}{*}{Ceuta}    & $\hat{\omega}$  & 0.98 (0.95 - 0.99) \\
                          & $\hat{q}$       & 0.28 (0.25 - 0.31) \\
\hline
\multirow{2}{*}{Extremadura}    & $\hat{\omega}$  & 0.98 (0.95 - 1.00) \\
                                & $\hat{q}$       & 0.40 (0.36 - 0.44) \\
\hline
\multirow{2}{*}{Galiza}     & $\hat{\omega}$  & 0.84 (0.33 - 0.98) \\
                            & $\hat{q}$       & 0.41 (0.35 - 0.56) \\
\hline
\multirow{2}{*}{Illes Balears}    & $\hat{\omega}$  & 0.98 (0.96 - 0.99) \\
                                  & $\hat{q}$       & 0.36 (0.33 - 0.39) \\
\hline
\multirow{2}{*}{Regi\'on de Murcia}    & $\hat{\omega}$  & 0.93 (0.45 - 0.98) \\
                                       & $\hat{q}$       & 0.46 (0.34 - 0.80) \\
\hline
\multirow{2}{*}{Madrid}      & $\hat{\omega}$  & 0.98 (0.96 - 0.99) \\
                             & $\hat{q}$       & 0.37 (0.34 - 0.40) \\
\hline
\multirow{2}{*}{Nafarroa}    & $\hat{\omega}$  & 0.99 (0.97 - 1.00) \\
                             & $\hat{q}$       & 0.30 (0.26 - 0.32)\\
\hline
\multirow{2}{*}{Euskadi}    & $\hat{\omega}$  & 0.99 (0.97 - 0.99) \\
                            & $\hat{q}$       & 0.27 (0.25 - 0.31) \\
\hline
\multirow{2}{*}{La Rioja}    & $\hat{\omega}$  & 0.98 (0.96 - 0.99) \\
                             & $\hat{q}$       & 0.31 (0.28 - 0.35) \\
\hline
\multirow{2}{*}{Melilla}    & $\hat{\omega}$  & 0.97 (0.95 - 0.99) \\
                            & $\hat{q}$       & 0.34 (0.31 - 0.37) \\
\hline
\multirow{2}{*}{Pa\'is Valenci\`a}    & $\hat{\omega}$  & 0.95 (0.40 - 0.98) \\
                                      & $\hat{q}$       & 0.46 (0.40 - 0.67) \\
\hline
\end{tabular}
\end{table}

\begin{table}[h]
\small\sf\centering
\caption{Impact of covariates for each Spanish CCAA. Reported values correspond to the median and percentiles 2.5\% and 97.5\% of the corresponding posterior distribution.\label{tab:ccaa_est2}}
\begin{tabular}{ccc}
\hline
CCAA & Parameter & Estimate (95\% CrI)\\
\hline
\multirow{2}{*}{Andaluc\'ia}  & $\hat{Vacc}$  & -1.71 (-2.66, -0.68) \\
                              & $\hat{Conf}$  & -1.67 (-2.31, -0.39) \\
\hline
\multirow{2}{*}{Arag\'on}    & $\hat{Vacc}$  & -1.06 (-1.36, -0.69) \\
                             & $\hat{Conf}$  & 0.76 (0.17, 1.43) \\
\hline
\multirow{2}{*}{Asturies}    & $\hat{Vacc}$  & -0.90 (-1.77, -0.63) \\
                             & $\hat{Conf}$  & 0.44 (0.11, 0.69) \\
\hline
\multirow{2}{*}{Cantabria}    & $\hat{Vacc}$  & -0.53 (-1.29, -0.25) \\
                              & $\hat{Conf}$  & -0.44 (-0.71, 0.002) \\
\hline
\multirow{2}{*}{Castilla y Le\'on}    & $\hat{Vacc}$  & -1.22 (-1.88, -0.60) \\
                                      & $\hat{Conf}$  & -0.84 (-1.33, -0.23) \\
\hline
\multirow{2}{*}{Castilla - La Mancha}    & $\hat{Vacc}$  & -0.80 (-1.11, -0.40) \\
                                         & $\hat{Conf}$  & 0.06 (-0.18, 0.44) \\
\hline
\multirow{2}{*}{Canarias}    & $\hat{Vacc}$  & -1.34 (-1.78, -1.06) \\
                              & $\hat{Conf}$ & -0.92 (-2.06, -0.29) \\
\hline
\multirow{2}{*}{Catalunya}    & $\hat{Vacc}$  & -1.51 (-1.97, -0.94) \\
                              & $\hat{Conf}$  & -0.25 (-0.52, 0.21) \\
\hline
\multirow{2}{*}{Ceuta}    & $\hat{Vacc}$  & -1.38 (-1.93, -0.84) \\
                          & $\hat{Conf}$  & 0.007 (-0.52, 0.34) \\
\hline
\multirow{2}{*}{Extremadura}    & $\hat{Vacc}$  & -0.72 (-1.30, -0.37) \\
                                & $\hat{Conf}$  & 1.45 (1.24, 1.83) \\
\hline
\multirow{2}{*}{Galiza}     & $\hat{Vacc}$  & -2.03 (-3.07, -1.34) \\
                            & $\hat{Conf}$  & -0.20 (-0.53, 0.18) \\
\hline
\multirow{2}{*}{Illes Balears}    & $\hat{Vacc}$  & -0.72 (-1.16, -0.34) \\
                                  & $\hat{Conf}$  & 0.74 (0.43, 1.01) \\
\hline
\multirow{2}{*}{Regi\'on de Murcia}    & $\hat{Vacc}$  & -1.97 (-3.07, -0.59) \\
                                       & $\hat{Conf}$  & 0.62 (-0.02, 1.36) \\
\hline
\multirow{2}{*}{Madrid}      & $\hat{Vacc}$  & -0.35 (-0.77, -0.07) \\
                             & $\hat{Conf}$  & 0.35 (-0.39, 0.59) \\
\hline
\multirow{2}{*}{Nafarroa}   & $\hat{Vacc}$  & -2.05 (-3.20, -1.33) \\
                            & $\hat{Conf}$  & -1.71 (-1.92, -0.53) \\
\hline
\multirow{2}{*}{Euskadi}    & $\hat{Vacc}$  & -0.10 (-0.24, 0.00) \\
                            & $\hat{Conf}$  & -0.42 (-0.69, -0.21) \\
\hline
\multirow{2}{*}{La Rioja}    & $\hat{Vacc}$  & -0.43 (-0.71, -0.22) \\
                             & $\hat{Conf}$  & -0.83 (-1.08, -0.35) \\
\hline
\multirow{2}{*}{Melilla}   & $\hat{Vacc}$  & -1.59 (-2.05, -0.93) \\
                           & $\hat{Conf}$  & -0.48 (-0.82, -0.11) \\
\hline
\multirow{2}{*}{Pa\'is Valenci\`a}    & $\hat{Vacc}$  & -1.70 (-2.64, -0.52) \\
                                      & $\hat{Conf}$  & 1.45 (1.24, 1.83) \\
\hline
\end{tabular}
\end{table}

%%%%%%%%%%%%%%%%%%%%%%%%%%%%%%%%%%%
%%                               %%
%% Additional Files              %%
%%                               %%
%%%%%%%%%%%%%%%%%%%%%%%%%%%%%%%%%%%

\end{backmatter}
\end{document}
