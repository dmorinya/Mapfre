% This is a template for BU-ECE Technical Report.
%
% Depending on report content and author preference, a BU-ECE report may be
% in one of the two following styles:
%
%   - genuine report based on ``report'' style, i.e., with chapters, much like
%     a thesis; can be single- or double-sided,
%
%   - report based on ``article'' style, i.e., with no chapters (only sections,
%     subsections, etc.), much like a journal or conference paper; can be
%     single- or double-sided.

% =====================================================================

%\documentclass[12pt]{report}          %Single-sided report style (chapters)
%\documentclass[12pt,twoside]{report}  %Double-sided report style (chapters)
%\documentclass[12pt]{article}         %Single-sided article style (no chapters)
\documentclass[12pt,twoside]{article} %Double-sided article style (no chapters)

\usepackage{bu_ece_report}
\usepackage[utf8]{inputenc}
\usepackage[spanish]{babel}
\usepackage{amsmath}
\usepackage{amsfonts}
\usepackage{amssymb}
\usepackage{makeidx}
\usepackage{graphicx}
\usepackage{lmodern}

% In case an adjustment of vertical or horizontal margins is needed
% due to particular LaTeX/dvips or OS installation, you can uncomment
% and edit the following definitions.
% -------------------------------------------------------------------
%\topmargin       0.00 in
%\oddsidemargin   0.50 in
%\evensidemargin  0.00 in

\begin{document}

% Definitions.
% ------------
\buecedefinitions%
        {SEGUROS DE SALUD TRAS LA PANDEMIA: MEDIR LA TRANSFORMACIÓN}
        {SEGUROS DE SALUD TRAS LA PANDEMIA. INFORME 1}
        {David Moriña, Amanda Fernández-Fontelo y Montserrat Guillén}
        {Junio 2022}
        {YYYY-NN} % Number of the report (four year digits and number)

% Box with title to fit the opening in the cover
% (adds an empty page in double-sided printing mode).
% ---------------------------------------------------
\buecereporttitleboxpage

% Title page
% (adds an empty page in double-sided printing mode).
% ---------------------------------------------------
\buecereporttitlepage

% Special page, e.g., if the report is restricted or
% to whom it is dedicated, etc., otherwise skip.
% (adds an empty page in double-sided printing mode).
% ---------------------------------------------------
%\bueceprefacepage{Here comes a special preface page. For example, if the report
%is restricted, then a suitable note can be included. This page can also be used
%to indicate to whom the document is dedicated, etc.}

% Report summary; max. 1 page.
% (adds an empty page in double-sided printing mode).
% ---------------------------------------------------
\pagenumbering{roman}
\setcounter{page}{1}
\buecereportsummary{Los seguros de salud constituyen uno de los ramos del seguro con mayor penetración en el mercado español y lo mismo ocurre en muchos de los países desarrollados. Su siniestralidad ha sufrido el impacto de la pandemia de Covid-19 en 2020 y 2021, especialmente en lo que se refiere a consultas y actos médicos que podían ser pospuestos. Las restricciones de movilidad supusieron un declive en la utilización del seguro por parte de los asegurados y una transformación de la interacción entre pacientes y sanitarios con una mayor utilización de la consulta telefónica. Este proyecto pretende estudiar cómo determinar si, (i) por el efecto de posponer visitas o (ii) por las secuelas de haber sufrido el virus (Covid persistente o efectos secundarios), va a producirse un exceso de siniestralidad y en su caso, cuándo se producirá. En este primer informe se describe la propuesta metodológica que se plantea para dar respuesta a las cuestiones de investigación que se abordarán en este proyecto.}

% Table of contents, list of figures and list of tables.
% ``\bueceemptypage'' adds empty page in double-sided
% printing mode and performs ``\clearpage'' in single-sided
% mode.
% ------------------------------------------------------
\tableofcontents\bueceemptypage
\listoffigures\bueceemptypage
\listoftables\bueceemptypage

% Switch on running headers for the report:
%   odd pages  - title (lowercase); if too long, use
%                the first few words followed by ``...'',
%   even pages - last names of the authors.
% -------------------------------------------------------
\buecereportheaders

% Introduction.
% -------------
\pagenumbering{arabic}
\setcounter{page}{1}

\section{Introducción}  % Article style
Las consecuencias derivadas de la pandemia provocada por el virus SARS-CoV-2 han afectado de manera contundente en muchos ámbitos de la actividad humana. Además de las consecuencias directas en relación con las defunciones provocadas por la enfermedad Covid-19 y la saturación de los sistemas de salud en numerosos países (incluyendo España y países de su entorno), en el año 2020 se ha detectado una disminución en el uso de los servicios del Sistema Público de Salud y de los servicios asociados a los seguros de salud privados. 
Los seguros de salud constituyen uno de los ramos del seguro con mayor penetración en el mercado español, con más de 12 millones de asegurados, más del 25\% de la población posee este tipo de cobertura y supera el 35\% en algunas zonas (\cite{UNESPA2020}) y lo mismo ocurre en muchos de los países desarrollados. Su siniestralidad ha sufrido el impacto en 2020 y 2021, especialmente en lo que se refiere a consultas y actos médicos que podían ser pospuestos. Las restricciones de movilidad supusieron un declive en la utilización del seguro por parte de los asegurados y una transformación de la interacción entre pacientes y sanitarios con una mayor utilización de la consulta telefónica. La pregunta es saber si, bien por el efecto de posponer visitas o bien por las secuelas de haber sufrido el virus (Covid persistente o efectos secundarios), va a producirse un exceso de siniestralidad en 2022 y los años sucesivos. Ya existen evidencias de una menor frecuencia de utilización de servicios de Salud en el Sistema Público en 2020, especialmente en lo relativo a cáncer (\cite{Cancer2020}), y se han impulsado protocolos para revertir esta situación en 2021 (\cite{MinisteriodeSanidad2021}). Sin embargo, es difícil determinar si la mayor frecuencia de siniestralidad que se observará será igual o superior a la infra-siniestralidad que se observó durante la pandemia. Para analizarlo en el Proyecto se está usando la metodología estadística de la infra-representación de casos como base de partida (\cite{Fernandez-Fontelo2016, FernandezFontelo2019}) y se extiende la misma al contexto de la sobre-representación. El planteamiento es determinar cómo es posible ver si el efecto rebote (i) se produce uniformemente o sólo para determinadas coberturas del seguro de salud, (ii) se da de forma homogénea o en función de características del asegurado o bien (iii) en qué momento del tiempo se recupera el nivel de utilización de prestaciones que se venía observando antes del incio de la pandemia. Van a realizarse muchos análisis sobre consecuencias en el gasto de salud a nivel del sistema público, pero las implicaciones para los seguros privados de salud también van a ser de interés. Sobre todo, es de esperar que, para monitorizar los efectos de la pandemia en los próximos años, se deban utilizar este tipo de aproximaciones, ya que no podrán compararse directamente grupos de población con características sociodemográficas diferentes, ni impactos en utilización de servicios de salud en general, ni prestaciones y coberturas diferentes. Entre las implicaciones podría hablarse también de una adecuación en la forma de aproximar la tarificación en este ramo, previendo rebotes de siniestralidad que aún no están siendo observados. Este Proyecto permitirá cuantificar el impacto de la pandemia en los seguros de salud, y cómo evaluarlo, estimando el grado de infra-uso que se dio en 2020 principalmente y usando técnicas avanzadas de ciencia de datos desarrolladas recientemente, así como nuevos métodos e innovaciones encaminadas a valorar el sobre-uso, con la finalidad de crear un sistema de seguimiento de la siniestralidad que detecte el cambio en la dinámica de utilización del seguro médico en particular y de cualquier otro ramo, en general. Aunque el Proyecto se centre en el desarrollo de la metodología y pueda ilustrarse mediante datos simulados o inspirados en el sistema público, se ensayará también en dados agregados y por lo tanto anonimizados, de cartera salud. Es razonable pensar que los resultados y conclusiones pueden ser generalizables a otros ramos y que sirvan para valorar posibles desigualdades entre países o regiones. 

Se estima que en el año 2020 las prestaciones totales rendidas por los seguros de salud han totalizado 6.300 millones de euros, de los cuales 6.200 millones se corresponden a las prestaciones de servicios médicos. En 2019, se estima que las prestaciones totales rendidas por este tipo de seguros han totalizado 6.600 millones de euros, de los cuales 6.500 millones se corresponden a las prestaciones de servicios médicos.

% Following sections, subsections, etc.
% -------------------------------------
\section{Series temporales clásicas}
Una serie temporal es una secuencia de $N$ observaciones, ordenadas cronológicamente, sobre una o diversas características (\cite{Pena2005}). Un proceso estocástico es una secuencia de variables aleatorias, ordenadas y equidistantes cronológicamente, referidas a una (proceso escalar) o varias (proceso vectorial) características de una unidad observable.

Al trabajar con un proceso estocástico es conveniente identificar si se trata de un proceso estacionario o no, es decir, si las propiedades estadísticas de cualquier secuencia finita del mismo son similares para cualquier segmentación. Esto implica que todas las variables aleatorias que componen el proceso están idénticamente distribuidas, independientemente del momento de tiempo en el cual han sido generadas. En este caso se considera que las propiedades son constantes a lo largo del tiempo y esto facilita la obtención de predicciones, ya que permite usar los valores constantes de la media para obtener observaciones futuras y generar intervalos de predicción. Por otro lado, si no se cumple esta condición el proceso estocástico es no estacionario. Las propiedades estadísticas de un proceso no estacionario son más complejas, pero se puede intentar modelar partiendo de una transformación sencilla (por ejemplo siguiendo la metodología de Box-Cox) con el objetivo de definir su estructura probabilística completa a partir de una única realización finita del mismo proceso.

Según el teorema de Wold, cualquier proceso estocástico ($y_t$) se piede representar como la suma de un proceso de ruido blanco ($\epsilon_t$) y uno puramente determinista ($z_t$):
\begin{equation}
 y_t = z_t + \sum_{j=1}^{\infty} \psi_j \cdot \epsilon_{t-j}
\end{equation}

La parte no determinista se puede escribir como el resultado de una transformación lineal del procéso de ruido blanco (proceso estocástico constante con esperanza cero y sin correlación estadística). Con base en esta definición, podemos encontrar 3 tipos diferentes de series temporales: Processos autoregresivos (AR), procesos de media móvil (MA) y mixtos (ARMA), las principales propiedades de los cuáles se detallan en las próximas secciones.

\subsection{Procesos autoregresivos}
Los procesos autoregresivos (AR) pueden describirse mediante la ecuación
\begin{equation}
y_t = \mu + \sum_{i=1}^p \phi_i \cdot y_{t-i} + \epsilon_t,
\end{equation}
donde $\epsilon_t$ sigue una distribución normal con media 0 y varianza $\sigma^2$ y $\phi_i$ son parámetros fijados. Como puede observarse de la expresión anterior, en un proceso con estructura AR la observación a tiempo $t$ depende de los $t-p$ valores anteriores y de un ruido blanco $\epsilon_t$. Las principales propiedades de este tipo de modelos son las siguientes:
\begin{itemize}
 \item Son invertibles
 \item El operador de retardo (lag) es estable si sus raíces están fuera del círculo unidad
 \item Es estacionario si y solo si es invertible y estable
 \item $E[y_t]=\mu$
 \item $Var[y_t]=E[y_t-\mu]^2=\gamma_0$
 \item Autocovarianza: $\gamma_k=E[(y_t-\mu)(y_{t-k}-\mu)]=\phi^k \cdot \gamma_0$; donde $k=1,2, \ldots$
 \item Función de autocorrelación simple (FAS): $\rho_k=\frac{\gamma_k}{\gamma_0}=\phi^k$
 \item Función de autocorrelación parcial (FAP): $\alpha_k=\frac{\rho_k-\rho_{k-1}^2}{1-\rho_{k-1}^2}$
\end{itemize}

\subsection{Procesos de media móvil}
Los procesos de media móvil (MA) pueden describirse mediante la siguiente ecuación:
\begin{equation}
y_t= \mu + \epsilon_t+\sum_{j=1}^q -\theta_j \cdot \epsilon_{t-j}
\end{equation}

En este caso, la observación a tiempo $t$ es una media ponderada de un proceso de ruido blanco $\epsilon_t$ y un retraso de $q$ periodos. Estos procesos satisfacen las siguientes propiedades:
\begin{itemize}
 \item Son estacionarios
 \item $E[y_t]=\mu$
 \item $Var[y_t]=E[y_t-\mu]^2=\sigma^2 (1+\sum_{i=1}^q \theta_q)$
\end{itemize}

\subsection{Procesos ARMA (AutoRegressive Moving Average)}
Los procesos autoregresivos y de media móvil pueden combinarse en un tipo de procesos más complejos (ARMA), que pueden expresarse de la forma siguiente:
\begin{equation}
y_t = \sum_{i=1}^p \phi_i \cdot y_{t-i} + \epsilon_t - \sum_{j=1}^q \theta_j \cdot \epsilon_{t-j}
\end{equation}
Estos procesos combinan las propiedades de los procesos autoregresivos y los procesos de media móvil, y resultan más flexibles para modelar series temporales que presentan estructuras complejas.

\section{Series temporales discretas}
La mayor parte de la bibliografía especializada y los paquetes de software estadístico más usados tratan las series temporales clásicas como las introducidas anteriormente, donde esencialmente las observaciones $x(t_i)$ son variables aleatorias continuas con una distribución normal. En cambio, a menudo las observaciones de una serie temporal son discretas o incluso categóricas, y en estos casos los modelos de series temporales clásicos pueden fallar o proporcionar estimaciones sesgadas e ineficaces. Una de las famílias de modelos más usadas en este contexto son los modelos INAR (INteger-valued AutoRegressive), una extensión de los modelos autoregresivos clásicos definidos anteriormente. En este caso, el modelo se define por la siguiente ecuación:

\begin{equation}
X_t = p_1 \circ X_{t-1} + \ldots + p_k \circ X_{t-k},
\end{equation}
donde $p_1, \ldots, p_k$ son parámetros fijos, el proceso es estacionario y $W_t$ y $X_{t-1}$ son independientes para todo $t$. Las innovaciones $W_t$ son independientes e idénticamente distribuidas, normalmente con una distribución de Poisson, aunque pueden considerarse otras distribuciones discretas para las innovaciones.

El operador $\circ$ de la ecuación anterior, que sustituye al producto usual en la definición de los modelos autoregresivos clásicos se llama $p$-\textit{thinning}, \textit{binomial subsampling} o \textit{binomial thinning}, definido por

\begin{equation}
p \circ X_t = \sum_{i=1}^{X_t} Y_i,
\end{equation}
donde $Y_i$ son variables aleatorias independientes e idénticamente distribuidas con distribución de Bernoulli con probabilidad de éxito $p$. Teniendo en cuenta las definiciones anteriores, queda claro que la distribución de $p \circ X_t$ es Binomial con parámetros $X_t$ y $p$. Un buen resumen de esta familia de modelos y otras alternativas para el análisis de series temporales discretas se pueden encontrar en \cite{McKenzie2003}.

\section{Nuevo modelo propuesto}  % Article style
%\chapter{Starting chapter} % Report style

Sea $X_n$ un proceso latente con estructura INAR(1) definida por: $X_n=\alpha \circ X_{n-1}+Z_n$, donde $\textrm{E}(X_n)=\mu_X$ y $\textrm{Var}(X_n)=\sigma_X^2$ representan la esperanza y la varianza de $X_n$, respectivamente. Asumamos, por ahora, que $Z_n \sim \textrm{Poisson}(\lambda)$. En cualquier caso, otras estructuras más apropiadas para el proceso latente pueden incorporarse dependiendo de la aplicación potencial (por ejemplo, puede generarse un proceso subyacente sobredisperso mediante innovaciones con distribución de Hermite de segundo orden, o procesos sin correlación temporal mediante modelos de Poisson o Hermite de segundo orden). Sea $Y_n$ un proceso observado y potencialmente sobre o infrareportado tal que: 
\begin{align}\label{eq0:modelfatthin}
 Y_n=\begin{cases} 
X_n &  1-\omega \\
\theta \Diamond X_n & \omega, \\
   \end{cases}
\end{align}

\noindent donde $\theta \Diamond X_n$ es el operador \textit{fattering-thinning} en el sentido que:
\begin{align}\label{eq1:fatteringthinning}
\theta \Diamond X_n|X_n=x_n=\sum_{j=1}^{x_n}W_j,
\end{align}

\noindent donde $\theta=(\phi_1,\phi_2)$ y $W_j$ son variables aleatorias independientes e idénticamente distribuidas definidas por la siguiente función de masa de probabilidad (pmf):

\begin{align}\label{eq2:pmfW}
\mathbb{P}(W_j=k|\phi_1,\phi_2)=\begin{cases} 
1-\phi_1-\phi_2 & \textrm{if } k=0  \\
\phi_1 & \textrm{if } k=1  \\
\phi_2 & \textrm{if } k=2  \\
0 & \textrm{otherwise}, \\
\end{cases}
\end{align}

\noindent Es importante remarcar que cuando $\phi_2=1$, el proceso no estaría ni sobre ni infrareportado; es decir, el proceso observado coincide con el proceso real. También es importante observar que una versión más restringida de (\ref{eq1:fatteringthinning}) resulta cuando $W_j \sim \textrm{Bernoulli}(2,\phi)$. Aunque la distribución mostrada en (\ref{eq2:pmfW}) es la elección mas directa que permite obtener sobrereporte, otras distribuciones sin soporte compacto pueden considerarse, por ejemplo Poisson, Geometrica, etc. \\
El operador en (\ref{eq1:fatteringthinning}) sigue una distribución Hermite de segundo orden con parámetros $\mu_X\phi_1$ y $\mu_X\phi_2$, como puede verse tomando la función generatriz de probabilidades (pgf), es decir:
\begin{align}
G_X(s)&=\textrm{e}^{\mu_X(s-1)}, \label{eq3:pgfpox}\\
G_W(s)&=(1-\phi_1-\phi_2)+\phi_1s+\phi_2s^2, \label{eq3:pgfw}\\
G_X\left(G_W(s)\right)&=\textrm{e}^{\mu_X\left((1-\phi_1-\phi_2)+\phi_1s+\phi_2s^2-1\right)}=\textrm{e}^{\mu_X\left(\phi_1(s-1)+\phi_2(s^2-1)\right)} \label{eq3:pgffath} ,
\end{align}
que es la función generatriz de probabilidades de una distribución de Hermite de segundo orden con parámetros $\mu_X\phi_1$ y $\mu_X\phi_2$. La esperanza y la varianza de este operador son respectivamente: $\textrm{E}=\left(\theta \Diamond X_n\right)=\mu_X\left(\phi_1+2\phi_2\right)$ y $\textrm{Var}=\left(\theta \Diamond X_n\right)=\mu_X\left(\phi_1+4\phi_2\right)$.

\subsection{Propiedades del modelo}

\medskip

\noindent La distribución marginal del proceso observado $Y_n$ es la mixtura siguiente de una distribución de Poisson y una Hermite:
\begin{align}\label{eq:mix}
Y_n=\begin{cases} 
\textrm{Poisson}(\mu_X) &  1-\omega, \\
\textrm{Hermite}\left(\mu_X\phi_1,\mu_X\phi_2\right) &  \omega. \\
\end{cases}
\end{align}
En cualquier caso, cuando las innovaciones del proceso latente INAR(1) son Hermite de segundo orden, la distribución en (\ref{eq:mix}) es una mixtura de dos componentes con distribución de Hermite de segundo orden. Distribuciones marginales más sencillas para el proceso observado $Y_n$ se obtienen cuando el proceso oculto $X_n$ sigue un modelo clásico de Poisson, o incluso un modelo Hermite de segundo orden.\\ 
La esperanza y la varianza de este proceso observado $Y_n$ son:
\begin{align*}
\textrm{E}(Y_n)&=(1-\omega)\mu_X+\omega\mu_X\left(\phi_2+2(1-\phi_1-\phi_2)\right)=\mu_X(1-\omega\left(1-(2(1-\phi_1)-\phi_2\right))).\\
\textrm{E}(Y_n^2)&=(1-\omega)(\sigma_X^2+\mu_X^2)+\omega \left(\mu_X(4(1-\phi_1)-3\phi_2)+\mu_X^2(2(1-\phi_1)-\phi_2)^2\right)\\&=\mu_X\left(1-\omega\left(1-\left(4(1-\phi_1)-\phi_2\right)\right)\right)+\mu_X^2\left(1-\omega\left(1-(2(1-\phi_1)-\phi_2)^2\right)\right),
\end{align*}
ya que $\mu_X=\sigma_X^2$, y 
\begin{align*}
\textrm{Var}(Y_n)&=\mu_X\left(1-\omega\left(1-\left(4\left(1-\phi_1\right)-\phi_2\right)\right)\right)+\mu_X^2\left(1-\omega\left(1-\left(2\left(1-\phi_1\right)-\phi_2\right)^2\right)\right)\\ &-\mu_X^2\left(1-\omega\left(1-\left(2\left(1-\phi_1\right)-\phi_2\right)\right)\right)^2\\&=\mu_X\left(1-\omega\left(1-\left(4\left(1-\phi_1\right)-\phi_2\right)\right)\right)+\mu_X^2\omega(1-\omega)\left(1-\left(2\left(1-\phi_1\right)-\phi_2\right)\right)^2. 
\end{align*}

\noindent Sea ${\bf 1}_n$ un indicador del estado de sobre o infrareporte en el sentido que ${\bf 1}_n \sim \textrm{Bernoulli}(\omega)$. Asumimos que los estados de sobre o infrareporte son independientes en el tiempo. 

\medskip

\noindent La función de autocovarianza de $Y_n$ se puede calcular de la manera siguiente:
\begin{align}\label{cov}
\textrm{Cov}\left(Y_n,Y_{n+k}\right)=\textrm{E}\left(Y_n,Y_{n+k}\right)-\textrm{E}(Y_n)\textrm{E}(Y_{n+1}). 
\end{align}

\noindent Supongamos que $\textbf{1}_n \sim \textrm{Bernoulli}(\omega)$ independientemente de $X_n$. Adicionalmente, asumimos que los estados de sobre o infrareporte son independientes. Por tanto:
\begin{align}\label{exp}
\textrm{E}\left(Y_n,Y_{n+k}\right)&=\textrm{E}\left(X_n(1-\textbf{1}_n),X_{n+k}(1-\textbf{1}_{n+k})\right)+\textrm{E}\left(X_n(1-\textbf{1}_n),\theta \Diamond X_{n+k}\textbf{1}_{n+k}\right) \nonumber \\ &+\textrm{E}\left(\theta \Diamond X_n\textbf{1}_n,X_{n+k}(1-\textbf{1}_{n+k})\right)+\textrm{E}\left(\theta \Diamond X_n\textbf{1}_n,\theta \Diamond X_{n+k}\textbf{1}_{n+k}\right),
\end{align}
donde
\begin{align*}
&\textrm{E}\left(X_n(1-\textbf{1}_n),X_{n+k}(1-\textbf{1}_{n+k})\right)=(1-\omega)^2\textrm{E}(X_n,X_{n+k}), \\
&\textrm{E}\left(X_n(1-\textbf{1}_n),\theta \Diamond X_{n+k}\textbf{1}_{n+k}\right)=(1-\omega)\omega(2(1-\phi_1)-\phi_2)\textrm{E}(X_n,X_{n+k}), \\
&\textrm{E}\left(\theta \Diamond X_n\textbf{1}_n,\theta \Diamond X_{n+k}\textbf{1}_{n+k}\right)=\omega^2(2(1-\phi_1)-\phi_2)^2\textrm{E}(X_n,X_{n+k}).
\end{align*}
Siguiendo los cálculos, 
\begin{align*}
\textrm{E}\left(Y_n,Y_{n+k}\right)&=(1-\omega)^2\textrm{E}(X_n,X_{n+k})+2(1-\omega)\omega(2(1-\phi_1)-\phi_2)\textrm{E}(X_n,X_{n+k})\\&+\omega^2(2(1-\phi_1)-\phi_2)^2\textrm{E}(X_n,X_{n+k})\\&=\textrm{E}(X_n,X_{n+k})\left((1-\omega)^2+2(1-\omega)\omega(2(1-\phi_1)-\phi_2)+\omega^2(2(1-\phi_1)-\phi_2)^2\right)\\&=\textrm{E}(X_n,X_{n+k})\left((1-\omega)+\omega(2(1-\phi_1)-\phi_2)\right)^2\\&=\textrm{E}(X_n,X_{n+k})(1-\omega(1-\left(2(1-\phi_1)-\phi_2\right)))^2.
\end{align*}
Finalmente, 
\begin{align*}
\textrm{Cov}\left(Y_n,Y_{n+k}\right)&=(\sigma_X^2\alpha^k+\mu_X^2)(1-\omega(1-\left(2(1-\phi_1)-\phi_2\right)))^2-\\
& - \mu_X^2(1-\omega\left(1-\left(2(1-\phi_1)-\phi_2\right)\right))^2\\&=\mu_X\alpha^k(1-\omega(1-\left(2(1-\phi_1)-\phi_2\right)))^2
\end{align*}
donde $\textrm{E}(X_n)=\mu_X=\sigma_X^2=\textrm{Var}(X_n)=$ y $\textrm{E}(X_n,X_{n+k})=(\sigma_X^2\alpha^k+\mu_X^2)$.

\medskip

\noindent La función de autocorrelación (ACF) del proceso observado se puede escribir como:
\begin{align*}
\textrm{Cor}\left(Y_n,Y_{n+k}\right)&=\frac{\alpha^k(1-\omega(1-\left(2(1-\phi_1)-\phi_2\right)))^2}{\left(1-\omega\left(1-\left(4\left(1-\phi_1\right)-\phi_2\right)\right)\right)+\mu_X\omega(1-\omega)\left(1-\left(2\left(1-\phi_1\right)-\phi_2\right)\right)^2}=\\
&=\alpha^k \textrm{c}(\alpha,\lambda,\omega,\phi_1,\phi_2).
\end{align*}

\medskip

\noindent Los cálculos anteriores pueden extenderse al caso en el cual los estados de sobre o infrareporte tienen correlación, usando una cadena de Markov de dos estados. Sea $R_n$ este nuevo modelo, que asume sobre o infrareporte y una estructura de dependencia entre estos estados de sobre o infrareporte. Como resultado, la esperanza y la varianza de este nuevo proceso, esto es, $\textrm{E}(R_n)$ y $\textrm{Var}(R_n)$ se mantienen de la manera presentada anteriormente. Esto es consecuencia del hecho que la distribución marginal de $R_n$ es la misma que la de $Y_n$, esto es, una mixtura de una distribución de Poisson y una distribución de Hermite de segundo orden (\ref{eq:mix}). 

\medskip

\noindent Sin embargo, esto no es cierto para las funciones de autocovarianza y autocorrelación de $R_n$, que son más complejas. En este sentido, la covarianza del proceso $R_n$ se puede calcular como se describe a continuación. 

\medskip

\noindent Recordemos que ${\bf 1}_n$ es un indicador del estado de sobre o infrareporte el sentido que ${\bf 1}_n \sim \textrm{Bernoulli}(\omega)$. Supongamos ahora que existe una estructura de dependencia entre estos estados que puede ser representada por una cadena de Markov binaria. Como se muestra en \cite{FernandezFontelo2019}, la probabilidad de transición ${\bf P}$ es: 
 \[
   {\bf P}=
  \left[ {\begin{array}{cc}
   1-p_{01} & p_{01} \\
   p_{01}\frac{1-\omega}{\omega} & 1-p_{01}\frac{1-\omega}{\omega} \\
  \end{array} } \right]. 
\]
Como se demuestra en \cite{FernandezFontelo2019}, la matriz de transición ${\bf P}^k$ puede escribirse en términos de $p_{01}^k$. Vale la pena también mencionar que el parámetro $p_{01}$ puede escribirse en términos del segundo valor propio de {\bf P}, esto es, $p_{01}=\omega(1-\lambda_2)$, donde $\lambda_2$ es el segundo valor propio de {\bf P}. 

Supongamos que los procesos ${\bf 1}_n$ y $R_n$ son mutuamente independientes. De forma similar a las expresiones (\ref{cov}) y (\ref{exp}), tenemos que:
{\small \begin{align*}
\textrm{E}\left(X_n(1-{\bf 1}_n),X_{n+k}({1-\bf 1}_n)\right)&=\textrm{E}\left(X_n,X_{n+k}\right)\textrm{P}({\bf 1}_n=0,{\bf 1}_{n+k}=0)=\textrm{E}\left(X_n,X_{n+k}\right)(1-\omega)(1-\omega(1-\lambda_2^k)),\\
\textrm{E}\left(X_n(1-{\bf 1}_n),\theta \Diamond X_{n+k}{\bf 1}_n\right)&=\textrm{E}\left(X_n,X_{n+k}\right)(2(1-\phi_1)-\phi_2)\textrm{P}({\bf 1}_n=0,{\bf 1}_{n+k}=1)\\&=\textrm{E}\left(X_n,X_{n+k}\right)(2(1-\phi_1)-\phi_2)\omega(1-\omega)(1-\lambda_2^k),\\
\textrm{E}\left(\theta \Diamond X_n{\bf 1}_n,\theta \Diamond X_{n+k}{\bf 1}_n\right)&=\textrm{E}\left(X_n,X_{n+k}\right)(2(1-\phi_1)-\phi_2)^2\textrm{P}(X_n=1, X_{n+k}=1)\\ &=\textrm{E}\left(X_n,X_{n+k}\right)(2(1-\phi_1)-\phi_2)^2\omega(1-(1-\omega)(1-\lambda_2^k)).
\end{align*}}

\noindent De los cálculos anteriores:
\begin{align*}
&\textrm{E}(R_n,R_{n+k})=\textrm{E}(X_n,X_{n+k})\left(\left(1-\omega\left(1-(2(1-\phi_1)-\phi_2)^2\right)\right)-\omega(1-\omega)(1-(2(1-\phi_1)-\phi_2))^2(1-\lambda_2^k)\right)=\\
&=(\alpha^k\sigma_X^2+\mu_X^2)\left(\left(1-\omega\left(1-(2(1-\phi_1)-\phi_2)^2\right)\right)- \omega(1-\omega)(1-(2(1-\phi_1)-\phi_2))^2(1-\lambda_2^k)\right)^2,
\end{align*}
y la función de autocovarianza toma la siguiente expresión:
\begin{align*}
&\textrm{Cov}(R_n,R_{n+k})=\textrm{E}(R_n,R_{n+k})-\textrm{E}(R_n)\textrm{E}(R_{n+k})=\\&=(\alpha^k\sigma_X^2+\mu_X^2)\left(\left(1-\omega\left(1-(2(1-\phi_1)-\phi_2)^2\right)\right)-\omega(1-\omega)(1-(2(1-\phi_1)-\phi_2))^2(1-\lambda_2^k)\right)\\ &- \mu_X^2\left(1-\omega\left(1-\left(2(1-\phi_1)-\phi_2\right)\right)\right)^2=\\&=(\alpha^k\sigma_X^2+\mu_X^2)\left(1-\omega\left(1-\left(2(1-\phi_1)-\phi_2\right)\right)\right)^2+(\alpha^k\sigma_X^2+\mu_X^2)\left(\lambda_2^k\omega(1-\omega)(1-\left(2(1-\phi_1)-\phi_2\right))^2\right)\\&-\mu_X^2\left(1-\omega\left(1-\left(2(1-\phi_1)-\phi_2\right)\right)\right)^2=\\&=\alpha^k\sigma_X^2\left(1-\omega\left(1-\left(2(1-\phi_1)-\phi_2\right)\right)\right)^2+\mu_X^2\lambda_2^k\omega(1-\omega)(1-\left(2(1-\phi_1)-\phi_2\right))^2+\\&+\sigma_X^2(\alpha\lambda)^k\omega(1-\omega)(1-\left(2(1-\phi_1)-\phi_2\right))^2. 
\end{align*}

\noindent Finalmente, la función de autocorrelación se puede escribir de la forma siguiente: 
{\scriptsize
\begin{align}
\textrm{Cor}(R_n,R_{n+k})=\frac{\alpha^k\left(1-\omega\left(1-\left(2(1-\phi_1)-\phi_2\right)\right)\right)^2+\mu_X\lambda_2^k\omega(1-\omega)(1-\left(2(1-\phi_1)-\phi_2\right))^2+(\alpha\lambda)^k\omega(1-\omega)(1-\left(2(1-\phi_1)-\phi_2\right))^2}{\left(1-\omega\left(1-\left(4\left(1-\phi_1\right)-\phi_2\right)\right)\right)+\mu_X\omega(1-\omega)\left(1-\left(2\left(1-\phi_1\right)-\phi_2\right)\right)^2}
\end{align}}

\subsection{Estimación de parámetros}
En este apartado nos centraremos en la descripción de dos métodos para estimar los parámetros de los modelos descritos en la sección anterior.
\subsubsection{Método basado en la función de verosimilitud}

\noindent Los parámetros del modelo se pueden estimar usando la función de verosimilitud. Para hacerlo, se puede usar el algoritmo \textit{forward}, ya que el cálculo directo de la verosimilitud no es posible de forma analítica. Más detalles sobre los cálculo de esta expresión de la función de verosimilitud basados en el algoritmo \textit{forward} se pueden encontrar en Fern\'andez-Fontelo {\it et al.} (2016, 2019). En el escenario considerado en este trabajo, las probabilidades \textit{forward} se pueden calcular usando la siguiente expresión:

\begin{align}
&\gamma_n\left(\boldsymbol{y}_{1:n},x_n\right)=& \sum_{x_{n-1}}\textrm{P}\left(Y_n=y_n|X_n=x_n\right)\textrm{P}\left(X_n=x_n|X_{n-1}=x_{n-1}\right) \gamma_{n-1}\left(\boldsymbol{y}_{1:n-1},x_{n-1}\right)
\end{align}

\noindent donde las probabilidades de emisión toman la siguiente expresión:
 \begin{align}
\textrm{P}(Y_n=y_n|X_n=x_n)=\begin{cases} 
0 &  \textrm{if} \quad y_n<x_n<\frac{y_n}{2} , \\
(1-\omega)+\omega \frac{x_n!}{n_0!n_1!n_2!}(1-\phi_1-\phi_2)^{n_0}\phi_1^{n_1}\phi_2^{n_2} &  \textrm{if} \quad y_n=x_n , \\
\omega \frac{x_n!}{n_0!n_1!n_2!}(1-\phi_1-\phi_2)^{n_0}\phi_1^{n_1}\phi_2^{n_2} &  \textrm{if} \quad  \frac{y_n}{2}\leq x_n <y_n, \\
1 & \textrm{if} \quad  y_n=0,
\end{cases}
\end{align}

\noindent siendo $n_0$, $n_1$ y $n_2$ el número de 0, 1 y 2, respectivamente, en una secuencia de longitud $x_n$ (por ejemplo, $\{W_1=w_1, W_2=w_2, \dots W_{x_n}=w_{x_n}\}$) con la restricción $\sum_{i=1}^{x_n}w_i=y_n$. Dado el valor observado de $y_n$ y un potencial valor de $x_n$, el número de posibles secuencias de 0, 1 y 2 manteniendo las condiciones anteriormente mencionadas es probablemente mayor que uno. En este caso, la suma de cada posible secuencia será la probabilidad de emisión de $Y_n=y_n|X_n=x_n$. Cabe observar que en el caso en que $W_i \sim \textrm{Binomial}(2,\phi)$, las probabilidades de emisión son como siguen:
 \begin{align}
\textrm{P}(Y_n=y_n|X_n=x_n)=\begin{cases} 
0 &  \textrm{if} \quad y_n<x_n<\frac{y_n}{2} , \\
(1-\omega)+\omega \binom{2x_n}{y_n}\phi^{y_n}(1-\phi)^{2x_n-y_n}&  \textrm{if} \quad y_n=x_n , \\
\omega \binom{2x_n}{y_n}\phi^{y_n}(1-\phi)^{2x_n-y_n} &  \textrm{if} \quad  \frac{y_n}{2}\leq x_n <y_n, \\
1 & \textrm{if} \quad  y_n=0.
\end{cases}
\end{align}

\noindent Las probabilidades de transición son las mismas que las descritas para escenarios de infrauso, ya que el proceso subyacente sigue el mismo modelo. Sin embargo, otras estructuras interesantes para el proceso subyacente pueden considerarse, como un proceso INAR(1) con innovaciones con distribución de Hermite de segundo orden o incluso un modelo sin correlaciones temporales (por ejemplo modelos de Poisson o de Hermite de segundo orden, etc).

\noindent Finalmente, la función de verosimilitud del proceso $\{Y_n\}$ puede calcularse recursivamente de la forma siguiente:
\begin{align}
\label{for:LF}
\textrm{P}(\boldsymbol{Y}_{1:N}=\boldsymbol{y}_{1:N})=\textrm{P}\left(Y_1=y_1,Y_2=y_2,\dots,Y_N=y_N\right)=\sum_{x_N=\frac{y_N}{2}}^{y_N}\gamma_N\left(\boldsymbol{y}_{1:N},x_N\right),
\end{align}

\noindent tomando $\textrm{P}(X_1=x_1)=\textrm{Poisson}(\frac{\lambda}{1-\alpha})$, o $\textrm{Hermite}()$ en el caso en el que las innovaciones del proceso INAR(1) siguen una distribución de Hermite de segundo orden. 

\section{Series temporales con el operador \textit{fattering-thinning}}

Supongamos una versión del modelo INAR(1) con la estructura siguiente: 
\begin{align}\label{eq:newINAR}
X_n=\theta \Diamond X_{n-1}+Z_n
\end{align}
dónde $X_n \sim \textrm{Poisson}(\lambda)$, y $\theta \Diamond X_{n-1}$ es el operador \textit{fattering-thinning} (\ref{eq1:fatteringthinning}) y (\ref{eq2:pmfW}). Tomando la función generatriz de probabilidades de $X_n$ (\ref{eq3:pgfpox}) y $\theta \Diamond X_{n-1}$ (\ref{eq3:pgffath}), la función generatriz de probabilidades de $Z_n$ tiene la expresión siguiente:
\begin{align}\label{eq:pgf1}
\textrm{G}_Z(s)=\frac{\textrm{e}^{\lambda(s-1)}}{\textrm{e}^{\lambda\left(\phi_2(s-1)+(1-\phi_1-\phi_2)(s^2-1)\right)}}=\textrm{e}^{\lambda\left((1-\phi_2)(s-1)+(\phi_1+\phi_2-1)(s^2-1)\right)}.
\end{align}
La expresión (\ref{eq:pgf1}) recuerda a la función generatriz de probabilidades de una distribución Hermite de segundo orden. Sin embargo, según Kemp y Kemp (1965), ambos parámetros de una distribución Hermite de segundo orden deben ser positivos. En este caso, $\phi_1+\phi_2-1<0$ y, por tanto, la expresión (\ref{eq:pgf1}) no es una función generatriz de probabilidades. Esto significa que la distribución marginal del proceso (\ref{eq:newINAR}) no puede ser Poisson. 

\medskip

\noindent Por otro lado, supongamos ahora que $X_n \sim \textrm{Hermite}(a_1,a_2)$, entonces 
\begin{align*}
\textrm{G}_X(\textrm{G}_W(s))&=\textrm{e}^{a_1\left(\phi_1+\phi_2s+(1-\phi_1-\phi_2)s^2-1\right)+a_2\left(\left(\phi_1+\phi_2s+(1-\phi_1-\phi_2)s^2\right)^2-1\right)}\\&=\textrm{e}^{\left(a_1(\phi_1^2-1)+a_2(\phi_1^2-1)\right)+\left(a_1\phi_2+2\phi_1\phi_2\right)s+\left(a_1(1-\phi_1-\phi_2)+a_2(2\phi_1(1-\phi_1-\phi_2)+\phi_2^2)\right)s^2}\\ &\textrm{e}^{2a_2\phi_2(1-\phi_1-\phi_2)s^3+a_2(1-(\phi_1+\phi_2))^2s^4},
\end{align*}
que es la función generatriz de probabilidades de una distribución Hermite de cuarto orden con parámetros $b_1=a_1\phi_2+2\phi_1\phi_2$, $b_2=a_1(1-\phi_1-\phi_2)+a_2(2\phi_1(1-\phi_1-\phi_2)+\phi_2^2)$, $b_3=2a_2\phi_2(1-\phi_1-\phi_2)$ y $b_4=a_2(1-(\phi_1+\phi_2))^2$. Recordando que $\textrm{G}_Z(s)=\textrm{G}_X(s)/\textrm{G}_X(\textrm{G}_W(s))$, la distribución marginal de $X_n$ no puede ser una Hermite de segundo ni de cuarto orden ya que el parámetro de $s^4$ es $-a_2(1-(\phi_1+\phi_2))^2<0$, y debería ser positivo para ser la pgf de una distribución de Hermite de cuarto orden.

\medskip

\noindent Otras distribuciones alternativas para $X_n$ como la binomial, binomial negativa y geométrica se han considerado. Sin embargo, ninguna de ellas produce una función generatriz de probabilidad adecuada, ni por tanto, una distribución conocida para las innovaciones del proceso descrito en (\ref{eq:newINAR}). 

\medskip

\noindent Desde otro punto de vista, sea $Z_n \sim \textrm{Poisson}(\lambda)$. Entonces, la distribución de $X_n$ se puede expresar como
\begin{align}
\frac{G_X(s)}{G_X(\phi_1+\phi_2s+(1-\phi_1-\phi_2)s^2)}=\textrm{e}^{\lambda(s-1)}.
\end{align}
No obstante, cualquier función generatriz de probabilidad conocida satisface esta igualdad. Esto significa que, cuando las innovaciones siguen una distribución de Poisson, la distribución marginal del modelo (\ref{eq:newINAR}) es desconocida. \\ De la misma manera, cuando $Z_n \sim \textrm{Hermite}(a_1, a_2)$, la distribución marginal de $X_n$ es desconocida ya que no existe ninguna función generatriz de probabilidad conocida que satisfaga la igualdad siguiente:
\begin{align}
\frac{G_X(s)}{G_X(\phi_1+\phi_2s+(1-\phi_1-\phi_2)s^2)}=\textrm{e}^{a_1(s-1)+a_2(s^2-1)}.
\end{align}

\section{Futuros desarrollos previstos}
El rendimiento del modelo descrito se valorará mediante un estudio de simulación exhaustivo, con el objetivo de comprobar que los métodos de estimación propuestos son capaces de recuperar los valores de los parámetros usados en cada situación simulada.

Adicionalmente, se han identificado 65 actos médicos (con un total de 1,581,271 usuarios atendidos en 2021) de entre todos los servicios asociados a una póliza de salud básica para los cuales el cambio de tendencia desde el infrauso provocado por las restricciones tomadas por el gobierno español y los gobiernos autonómicos en 2020 para combatir la Covid-19 hasta el sobreuso esperado en 2021, derivado también de las consecuencias directas e indirectas de la pandemia. Para cada uno de estos actos médicos, se dispondrá de los campos siguientes en el horizonte temporal 2019-2021:
\begin{itemize}
 \item Identificador del paciente
 \item Fecha del siniestro / acto
 \item Especialidad realizadora
 \item Especialidad acto
 \item Grupo del acto
 \item Servicio concertado realizador
 \item Provincia del SS.CC. realizador
 \item Sexo
 \item Edad
 \item Número de póliza
 \item Fecha inicio / finalización póliza
\end{itemize}

Además, para poder contextualizar en cada caso, se dispondrá del número de pólizas vigentes en cada mes del periodo considerado.

%Section with a figure (Fig.~\ref{fig:example}).


% Plots (PostScript files) are included through the ``figure'' environment.
% For more complicated figures use the minipage commaned (see LaTeX manual).
% --------------------------------------------------------------------------
%\begin{figure}[htb]
%
%  \begin{minipage}[t]{0.49\linewidth}\centering
%    \centerline{\epsfig{figure=figures/regbsdcod.eps,width=8.0cm}}
%    \Ovalbox{\vbox to 1.5in{\vfill\hbox{\vtop{\hsize=2.5in\hfill}\hfill}\vfill}}
%    \medskip
%    \centerline{(a)}
%  \end{minipage}\hfill
%
%  \begin{minipage}[t]{0.49\linewidth}\centering
%    \centerline{\epsfig{figure=figures/regbsdcod.eps,width=8.0cm}}
%    \Ovalbox{\vbox to 1.5in{\vfill\hbox{\vtop{\hsize=2.5in\hfill}\hfill}\vfill}}
%    \medskip
%    \centerline{(b)}
%  \end{minipage}

%  \bigskip

%  \begin{minipage}[t]{0.49\linewidth}\centering
%    \centerline{\epsfig{figure=figures/regbsdcod.eps,width=8.0cm}}
%    \Ovalbox{\vbox to 1.5in{\vfill\hbox{\vtop{\hsize=2.5in\hfill}\hfill}\vfill}}
%    \medskip
%    \centerline{(c)}
%  \end{minipage}\hfill
%
%  \begin{minipage}[t]{0.49\linewidth}\centering
%    \centerline{\epsfig{figure=figures/regbsdcod.eps,width=8.0cm}}
%    \Ovalbox{\vbox to 1.5in{\vfill\hbox{\vtop{\hsize=2.5in\hfill}\hfill}\vfill}}
%    \medskip
%    \centerline{(d)}
%  \end{minipage}
%
%  \caption{Block diagram: (a) one; (b) two; (c) three, and (d) four.}
%  \label{fig:example}
%\end{figure}

% Bibliography.
% -------------
\parskip=0pt
\parsep=0pt
\bibliographystyle{ieeetrsrt}

% Important: substitute your BiBTeX (*.bib) files below.
% ------------------------------------------------------
\bibliography{report}

\end{document}
